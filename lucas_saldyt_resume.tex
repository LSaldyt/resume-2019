% resume.tex
%
% (c) 2002 Matthew Boedicker <mboedick@mboedick.org> (original author) http://mboedick.org
% (c) 2003-2007 David J. Grant <davidgrant-at-gmail.com> http://www.davidgrant.ca
% (c) 2019 Lucas Saldyt <lucassaldyt@gmail.com> https://github.com/LSaldyt
%
% This work is licensed under the Creative Commons Attribution-ShareAlike 3.0 Unported License. To view a copy of this license, visit http://creativecommons.org/licenses/by-sa/3.0/ or send a letter to Creative Commons, 171 Second Street, Suite 300, San Francisco, California, 94105, USA.

% Thank you for having ONE tex file. The last time I had to edit a LaTeX resume, the intern in question had multiple practically-nested 
% tex files and it was a mess for me to sort out. An unrelated general comment: I always advise people to keep a big master copy of their resume (more
% like a CV, really) and derive one and two page versions from that - then you don't lose anything important or cool along the way.
% I didn't compile this, just edited it, so I reserve the right to have made typos.
% -JS

\documentclass[letterpaper,11pt]{article}

%-----------------------------------------------------------
%Margin setup

\setlength{\voffset}{0.05in}
\setlength{\paperwidth}{8.5in}
\setlength{\paperheight}{11in}
\setlength{\headheight}{0in}
\setlength{\headsep}{0in}
\setlength{\textheight}{10in}
\setlength{\topmargin}{-0.25in}
\setlength{\textwidth}{7in}
\setlength{\topskip}{0in}
\setlength{\oddsidemargin}{-0.25in}
\setlength{\evensidemargin}{-0.25in}
%-----------------------------------------------------------
\usepackage{shading}
\usepackage{changepage}
\pagestyle{empty}
\raggedbottom
\raggedright
\setlength{\tabcolsep}{0in}

%-----------------------------------------------------------
%Custom commands
\newcommand{\indented}[1]{
    \begin{adjustwidth}{1cm}{}
        #1
    \end{adjustwidth}
}
\newcommand{\sectionline}{
    \noindent\rule[0.5ex]{\linewidth}{0.5pt}
}

\newcommand{\resitem}[1]{\item #1 \vspace{-3pt}}
\newcommand{\resheading}[1]{
    {\large \textbf{#1}}
    \sectionline
    %{\large \parashade[.8]{sharpcorners}{\textbf{\underline{#1}}}} %\vphantom{p\^{E}}}}}
}
\newcommand{\colfill}{@{\extracolsep{\fill}}}
\newcommand{\ressubheading}[4]{
\begin{tabular*}{6.5in}{l\colfill r}
		\textbf{#1} & #2 \\
		\textit{#3} & \textit{#4} \\
\end{tabular*}\vspace{-6pt}}

%-----------------------------------------------------------


\input{glyphtounicode}
\pdfgentounicode=1
\begin{document}
\begin{center}
\textbf{\Large Lucas Saldyt}
\end{center}
\begin{tabular*}{7in}{l \colfill r}
lucassaldyt@gmail.com & http://github.com/LSaldyt\\
% \hline
505-506-1245 & Mesa, Arizona \\
\end{tabular*}
\\

\vspace{0.1in}

\resheading{Education}
\begin{itemize}
\item
    \ressubheading{Arizona State University: Barrett, The Honors College}{Tempe, Arizona}{Bachelor of Science in Computer Science, GPA: 3.7}{Sep. 2017 - Current}
\item
    \ressubheading{MIT Open Courseware}{Online}{Quantum Computation, AI, and CS courses}{}
	%\begin{itemize}
			% If you find yourself needing space, specific courses are the first to go, though you'd have to find somewhere to list the experience. -JS
            % \resitem{Data Structures and Algorithms (Demaine), Quantum Algorithmic Complexity (Aaronson), Quantum Mechanics (Zwiebach), Artificial Intelligence (Winston), Artificial General Intelligence (Fridman), Society of Mind (Minsky), Information Theory (Lloyd)}
	%\end{itemize}

\end{itemize}

\vspace{0.1in}

\resheading{Experience}
\begin{itemize}
 \item
 	% Don't bother with the acronym. -JS
     \ressubheading{National Aeronautics and Space Administration}{Cape Canaveral, Florida}{Software Engineering Intern}{Jun. 2019 - Aug. 2019}
 	\begin{itemize}
 		% The next line is nearly a game of buzzword bingo. -JS
 		\resitem{Worked on class A, safety-critical, human rated spaceflight ground control software by participating in the full software development lifecycle and using agile processes}
        \resitem{Created, benchmarked, and optimized verification/validation software for launch control tests}
 		\resitem{Independently prototyped original display profile saving system for launch control engineers}
 	\end{itemize}
 \item
    \ressubheading{Sandia National Laboratories (Dr. Erik Nielsen)}{Albuquerque, New Mexico}{Quantum Computation Intern}{Jun. 2015 - Sep. 2018}
 	\begin{itemize}
        \resitem{Developed high-fidelity quantum benchmarking (Gate Set Tomography) software}
 		\resitem{Created distributed high-performance simulation, verification, and data analysis software}
 		\resitem{Assisted in publishing papers in quantum benchmarking}
 	\end{itemize}
 \item
     \ressubheading{Los Alamos National Laboratories (Dr. Scott Pakin)}{Albuquerque, New Mexico}{Quantum Computation Shadow}{Apr. 2017}
 	\begin{itemize}
 		\resitem{Benchmarked the knapsack problem on LANL's DWave annealer and IBM's machines}
 	\end{itemize}
 \item
    \ressubheading{ASU Complex Systems Research (Dr. Yun Kang)}{Tempe, Arizona}{Mathematics Research Assistant}{Oct. 2018 - Current}
 	\begin{itemize}
 		\resitem{Unique math/computer modeling and visualization of ant nest choice and alarm propagation}
 		% I would pull any publications out into their own separate section, especially if that's a first authorship -JS
        \resitem{Author of a computation biology paper on alarm propagation, published in PNAS}
 	\end{itemize}
 \item
     \ressubheading{Fulton Undergraduate Research Initiative (Dr. Ajay Bansal)}{Tempe, Arizona}{Machine Learning Researcher}{Sep. 2018 - Jun. 2019}
 	\begin{itemize}
        \resitem{Developed Qurry, a quantum programming language}
        \resitem{Machine learning research, focused around Kolmogorov complexity and program learning}
 	\end{itemize}
 \item
    \ressubheading{The Fluid Analogies Research Group (Dr. Alexandre Linhares)}{Remote (paid)}{Cognitive Science Research Assistant}{Oct. 2016 - Sep. 2018}
 	\begin{itemize}
 		\resitem{Revitalized of Douglas Hofstadter's ``copycat'' cognitive model}
 		\resitem{Statistical analysis/visualization and comparison of various models to human data}
 	\end{itemize}
 \item
     \ressubheading{Unitary Fund}{Remote (paid)}{Quantum Software Researcher}{Jun. 2018 - Current}
 	\begin{itemize}
            \resitem{Prototyping of a quantum programming language, called ``Qurry''}
            \resitem{Presented in Brussels, Belgium at the FOSDEM Quantum Computing Conference}
 	\end{itemize}
\end{itemize}

\resheading{Skills}
\begin{description}
    \item[Programming Languages:] 
        Python, C++, Java, Bash, Clojure (LISPs), Haskell, C, MATLAB, R, Fortran
    \item[Applications:]
        Vim, \LaTeX, Git, MPI, Supercomputing (Slurm), Jupyter Notebook, Autodesk Design 
    \item[Operating Systems:]
        Linux, MacOS X, Windows
    \item[Natural Languages:] 
        English, Ukranian, Spanish
\end{description}
\end{document}
